\documentclass[12pt]{article}
\usepackage{amsmath}
\usepackage{amssymb}
\usepackage{parskip}
\usepackage{graphicx}
\usepackage{bm}


\newcommand{\R}{\mathbb{R}} 

\begin{document}
\title{Stuff I Should Know for Bayesian Statistics}
\author{D. Aaron Hillegass \\hillegass@gatech.edu}
\maketitle

\section*{Normal Distribution}

A normal distribution has two parameters: the mean ($\theta$) and the variance ($\sigma^2$). $x$ can go from $-\infty$ to $\infty$. Its density is given by
\begin{equation*}
p(x) \propto e^{-\frac{(x - \theta)^2}{2\sigma^2}}
\end{equation*}

For that to integrate to 1, the normalizing constant is
\begin{equation*}
\frac{1}{\sqrt{2\pi\sigma^2}}
\end{equation*}

The conjugate prior of the mean of a normal distribution is another normal distribution. Specifically, if $x_1, \dots, x_n$, are identical and independent with a normal distribution $\mathcal{N}(\theta, \sigma^2)$ and the prior is $\theta \sim  N(\theta_0, \sigma_0^2)$,
\begin{equation*}
\theta | x_1,\ldots,x_n \sim \mathcal{N}\left( \frac{\frac{\sigma^2}{n}\theta_0 + \sigma_0^2\bar{x}}{\frac{\sigma^2}{n}+\sigma_0^2},\left( \frac{n}{\sigma^2} + \frac{1}{\sigma_0^2} \right)^{-1} \right)
\end{equation*}

The conjugate prior of the variance of a normal distribution is an inverse gamma distribution.

\section*{Gamma Distribution}

The gamma distribution has two parameters: $\alpha$ and $\beta$. $x$ can go from $0$ to $\infty$ Its density is given by
\begin{equation*}
p(x) \propto   x^{\alpha-1} e^{-\beta x} \quad \text{ for } x > 0 \text{ and } \alpha, \beta > 0,
\end{equation*}

For that to integrate to 1, the normalizing constant is
\begin{equation*}
\frac{ \beta^\alpha}{\Gamma(\alpha)}
\end{equation*}

The mean of the distribution is
\begin{equation*}
E[x] = \frac{ \alpha}{\beta}
\end{equation*}

If $\alpha \geq 1$, the mode is 
\begin{equation*}
\frac{ \alpha - 1}{\beta}
\end{equation*}


\section*{Inverse Gamma Distribution}

An inverse gamma distribution has two parameters: $\alpha$ and $\beta$. $x$ can go from $0$ to $\infty$. Its density is given by
\begin{equation*}
p(x) \propto 
(1/x)^{\alpha + 1}e^{\left(-\frac{\beta}{x}\right)}
\end{equation*}

For that to integrate to 1, the normalizing constant is
\begin{equation*}
\frac{\beta^\alpha}{\Gamma(\alpha)}
\end{equation*}

If $\alpha > 1$, the mean of the distribution is
\begin{equation*}
E[x] = \frac{ \beta}{\alpha - 1} 
\end{equation*}

The mode is 
\begin{equation*}
\frac{ \beta}{\alpha + 1}
\end{equation*}

\section*{Binomial Distribution}

The binomial distribution is a discrete distribution that takes one parameter: $\pi$ is the probability of one success in one try. The probability of getting exactly $k$ successes in $n$ trials is given by
\begin{equation*}
p(k) = \binom{n}{k}\pi^k(1-\pi)^{n-k}
\end{equation*}

The mean is $n\pi$.

The conjugate prior of $\pi$ is the Beta distribution.

\section*{Beta Distribution}

The beta distribution takes two parameters: $\alpha$ and $\beta$. Its density is given by
\begin{equation*}
p(x) \propto  x^{\alpha-1}(1-x)^{\beta-1}
\end{equation*}

For that to integrate to 1, the normalizing constant is
\begin{equation*}
\frac{1}{Beta(\alpha,\beta)}
\end{equation*}

The mode is
\begin{equation*}
\frac{\alpha - 1} {\alpha + \beta - 2}
\end{equation*}

\section*{Poisson Distribution}

The Poisson Distribution is a discrete distribution that takes one parameter: $\lambda$. Its distribution is given by
\begin{equation*}
p(x) = \frac{\lambda^x}{x!} \hspace{5mm}  x \in \lbrace 0, 1, \ldots\rbrace
\end{equation*}

To normalize it, the constant is
\begin{equation*}
e^{-\lambda}
\end{equation*}

$\lambda$ is both its mean and its variance.

The congugate prior for $\lambda$ is the gamma distribution

\section*{Pareto Distribution}

The Type 1 Pareto Distribution has two parameters $\alpha$ and $x_\mathrm{m}$, where $x_\mathrm{m}$ is the minimum possible value.  Both  $x_\mathrm{m}$ and $\alpha$ must be positive. Its density is
\begin{equation*}
p(x)= \begin{cases} \frac{\alpha x_\mathrm{m}^\alpha}{x^{\alpha+1}} & x \ge x_\mathrm{m}, \\ 0 & x < x_\mathrm{m}. \end{cases}
\end{equation*}

The mean of the distribution is
\begin{equation*}
\operatorname{E}(X)= \begin{cases} \infty & \alpha\le 1, \\
\frac{\alpha x_\mathrm{m}}{\alpha-1} & \alpha>1.
\end{cases}
\end{equation*}

Its cumulative distribution function is
\begin{equation*}
F(x) = \begin{cases}
1-\left(\frac{x_\mathrm{m}}{x}\right)^\alpha & x \ge x_\mathrm{m}, \\
0 & x < x_\mathrm{m}.\end{cases}
\end{equation*}

\section*{Gamma Function}

The gamma function is a generalization of factorial. That is, if $n$ is a positive integer
\begin{equation*}
\Gamma(n) = (n-1)!
\end{equation*}
But the gamma function is defined:
\begin{equation*}
\Gamma(z) = \int_0^\infty x^{z-1} e^{-x}\,dx
\end{equation*}

\section*{Beta Function}

The beta function is 
\begin{equation*}
Beta(x,y) = \int_0^1t^{x-1}(1-t)^{y-1}\,dt = \frac{\Gamma(x)\,\Gamma(y)}{\Gamma(x+y)}
\end{equation*}

\section*{Choose Function}

The choose function is:
\begin{equation*}
\binom{n}{k} = \frac{n!}{k! (n-k)!}
\end{equation*}


\section*{Linear Regression}

If you are trying to use $p$ inputs ($x_1, \ldots, x_p$) to predict an ouput $y$, you could create a vector of coefficients $(\beta_0, \ldots, \beta_n)$ such that
\begin{equation*}
y \approx \beta_0 + \beta_1 x_1 + \ldots + \beta_p x_p
\end{equation*}

So how do you figure out what the $\beta$ is? You use the $n$ data points you have: The column vector $\boldsymbol{y} = (y_1, y_2, \ldots, y_n)'$ is the outcomes. The matrix $\boldsymbol{X}$ is the inputs.  Row $i$ of $\boldsymbol{X}$ starts with 1 and has the inputs $(x_{i,1}, x_{i,2},\ldots,x_{i, p})$ that correspond to $y_i$. Thus $X$ has $n$ rows and $p+1$ columns.

Given the data you have, the sum of the squares of the errors for a column vector $\beta$ (of length $p + 1$) is given by 
\begin{equation*}
\boldsymbol{y}^T \boldsymbol{y} - 2 \beta^T \boldsymbol{X}^T \boldsymbol{y} + \beta^T \boldsymbol{X}^T \boldsymbol{X} \beta
\end{equation*}

To minimize that, you set $\beta$ to be:
\begin{equation*}
\beta = \left( \boldsymbol{X}^T \boldsymbol{X} \right)^{-1} \boldsymbol{X}^T \boldsymbol{y}
\end{equation*}

\section*{Logarithmic Identities}

\begin{equation*}
\log{\left( e^x \right)} = x \hspace{10mm} e^{\log{x}} = x
\end{equation*}

\begin{equation*}
\log{\left(  ab \right)} = \log a + \log b \hspace{10mm}  \log{\left(\frac{a}{b} \right)} = \log a - \log b
\end{equation*}

\begin{equation*}
\log{\left(  a^b \right)} = b\log a \hspace{10mm}  \log{\left(\sqrt[b]{a} \right)} = \frac{\log{a}}{b}
\end{equation*}

\begin{equation*}
a^{\log{b}} = b^{\log{a}}  \hspace{10mm}  x^{\frac{\log(a)}{\log(x)}} = a
\end{equation*}

\begin{equation*}
{d \over dx} \log x = {1 \over x } \hspace{10mm} \log x = \int_1^x \frac {1}{t} dt
\end{equation*}

\end{document}
